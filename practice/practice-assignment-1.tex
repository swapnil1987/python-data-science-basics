\documentclass[11pt]{article}
\usepackage[margin=1in]{geometry}
\usepackage{amsmath}
\usepackage{enumitem}
\usepackage{fancyhdr}
\usepackage{listings}
\usepackage{xcolor}

\lstset{
    language=Python,
    basicstyle=\ttfamily\small,
    keywordstyle=\color{blue},
    stringstyle=\color{red},
    commentstyle=\color{gray},
    breaklines=true,
    frame=single,
    numbers=left,
    numberstyle=\tiny\color{gray}
}

\pagestyle{fancy}
\fancyhf{}
\rhead{Python Programming Assignment}
\lhead{Economics Data Science}
\rfoot{Page \thepage}

\title{\textbf{Python Programming Assignment} \\ Control Flow and Functions}
\author{Swapnil Singh}
\date{}

\begin{document}

\maketitle

\section*{Instructions}
 Ensure your code is well-commented and follows proper style conventions. Test your functions with multiple examples to verify correctness.

\vspace{0.3cm}

\section{Control Flow Problems}

\subsection*{Problem 1: Tax Bracket Calculator (Easy)}
Write a program that calculates income tax based on the following simplified tax brackets:
\begin{itemize}
    \item Income up to \$10,000: 0\% tax
    \item Income from \$10,001 to \$40,000: 10\% tax
    \item Income from \$40,001 to \$85,000: 20\% tax
    \item Income above \$85,000: 30\% tax
\end{itemize}

Use conditional statements to determine which bracket applies and calculate the total tax owed. The tax should be applied progressively (e.g., someone earning \$50,000 pays 0\% on the first \$10,000, 10\% on the next \$30,000, and 20\% on the remaining \$10,000).

\subsection*{Problem 2: GDP Growth Classification (Easy)}
Write a program that takes a country's annual GDP growth rate (as a percentage) and classifies it as:
\begin{itemize}
    \item ``Recession'' if growth $<$ 0
    \item ``Slow Growth'' if 0 $\leq$ growth $<$ 2
    \item ``Moderate Growth'' if 2 $\leq$ growth $<$ 4
    \item ``Strong Growth'' if growth $\geq$ 4
\end{itemize}

Use if-elif-else statements to print the appropriate classification.

\subsection*{Problem 3: Price Index Calculator (Medium)}
Write a program that calculates the Consumer Price Index (CPI) for a basket of goods over 5 years. Given:
\begin{itemize}
    \item A list of prices for Year 1 (base year): \texttt{[2.5, 1.8, 3.2, 0.9, 4.1]}
    \item A list of prices for Years 2-5 (create your own reasonable values)
\end{itemize}

Use a for loop to calculate the CPI for each year using the formula:
$$\text{CPI} = \frac{\sum \text{Current Year Prices}}{\sum \text{Base Year Prices}} \times 100$$

Print the CPI for each year with appropriate labels.

\subsection*{Problem 4: Compound Interest with Conditions (Medium)}
Write a program that calculates the future value of an investment with the following conditions:
\begin{itemize}
    \item Initial investment: \$10,000
    \item Base annual interest rate: 5\%
    \item If the investment balance exceeds \$15,000, the interest rate increases to 6\%
    \item If the balance exceeds \$20,000, the interest rate increases to 7\%
\end{itemize}

Use a while loop to calculate the balance year by year until it reaches at least \$25,000. Print the year and balance for each iteration, and show how many years it takes to reach the target.

\section{Function Problems}

\subsection*{Problem 5: Elasticity Calculator (Easy)}
Write a function \texttt{price\_elasticity(p1, q1, p2, q2)} that calculates the price elasticity of demand using the midpoint method:
$$E_d = \frac{(Q_2 - Q_1) / [(Q_2 + Q_1)/2]}{(P_2 - P_1) / [(P_2 + P_1)/2]}$$

The function should:
\begin{itemize}
    \item Take four parameters: initial price, initial quantity, new price, new quantity
    \item Return the elasticity value
    \item Include a docstring explaining what the function does
\end{itemize}

Test your function with: $P_1 = 10$, $Q_1 = 100$, $P_2 = 12$, $Q_2 = 80$.

\subsection*{Problem 6: Descriptive Statistics Function (Medium)}
Write a function \texttt{describe\_data(data)} that takes a list of numerical values and returns a dictionary containing:
\begin{itemize}
    \item \texttt{``mean''}: the arithmetic mean
    \item \texttt{``median''}: the median value
    \item \texttt{``min''}: the minimum value
    \item \texttt{``max''}: the maximum value
    \item \texttt{``range''}: the range (max - min)
\end{itemize}

Do not use external libraries like NumPy. Implement the calculations using basic Python operations and list methods.

Test your function with the dataset: \texttt{[23, 45, 12, 67, 34, 89, 23, 56, 78, 45]}.

\subsection*{Problem 7: Present Value Calculator (Medium)}
Write a function \texttt{present\_value(future\_cash\_flows, discount\_rate)} that:
\begin{itemize}
    \item Takes a list of future cash flows and an annual discount rate
    \item Calculates the present value of each cash flow using: $PV = \frac{CF_t}{(1 + r)^t}$
    \item Returns both a list of individual present values and the total NPV
\end{itemize}

Use a default parameter value of 0.05 (5\%) for the discount rate.

Test with cash flows: \texttt{[1000, 1500, 2000, 2500, 3000]} over 5 years.

\subsection*{Problem 8: Monte Carlo Simulation Function (Hard)}
Write a function \texttt{simulate\_returns(initial\_investment, years, simulations=1000)} that:
\begin{itemize}
    \item Simulates investment returns over a specified number of years
    \item Assumes annual returns are randomly drawn from a normal distribution with mean 7\% and standard deviation 15\%
    \item Runs the specified number of simulations
    \item Returns a dictionary with:
    \begin{itemize}
        \item \texttt{``mean\_final\_value''}: average final portfolio value across all simulations
        \item \texttt{``median\_final\_value''}: median final portfolio value
        \item \texttt{``percentile\_5''}: 5th percentile (worst case in 95\% of scenarios)
        \item \texttt{``percentile\_95''}: 95th percentile (best case in 95\% of scenarios)
    \end{itemize}
\end{itemize}

Hint: Use \texttt{import random} and \texttt{random.gauss(mean, std\_dev)} to generate random returns.

Test with: \$10,000 initial investment over 10 years with 1000 simulations.

\section*{Bonus Challenge (Optional)}
Combine multiple concepts: Write a function that simulates a simple market equilibrium using an iterative process. Start with an initial price guess, calculate supply and demand at that price using linear functions, adjust the price based on excess demand/supply, and repeat until equilibrium is reached (supply $\approx$ demand within a tolerance). Your function should return the equilibrium price and quantity.

\end{document}