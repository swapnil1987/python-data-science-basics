\documentclass[11pt]{article}
\usepackage[margin=1in]{geometry}
\usepackage{amsmath}
\usepackage{enumitem}
\usepackage{fancyhdr}
\usepackage{listings}
\usepackage{xcolor}

\lstset{
    language=Python,
    basicstyle=\ttfamily\small,
    keywordstyle=\color{blue},
    stringstyle=\color{red},
    commentstyle=\color{gray},
    breaklines=true,
    frame=single,
    numbers=left,
    numberstyle=\tiny\color{gray}
}

\pagestyle{fancy}
\fancyhf{}
\rfoot{Page \thepage}

\title{\textbf{Python Programming Assignment} \\ Using \texttt{pandas}}
\author{Swapnil Singh}
\date{}

\begin{document}

\maketitle

\section*{Instruction}
 Ensure your code is well-commented and follows proper style conventions. You will be able to do this practice by using tools which were discussed during the class.



In this assignment you will work with two data files (look at the solution file):

\begin{itemize}
    \item \texttt{students.txt}
    \item \texttt{publications.txt}
\end{itemize}


\section*{Part A: Loading and Inspecting the Data}

\begin{enumerate}
    \item Import pandas and check the installed version.
    \begin{enumerate}
        \item Import pandas with the usual alias.
        \item Print the value of \verb|pd.__version__|.
    \end{enumerate}

    \item Load both data files into DataFrames.
    \begin{enumerate}
        \item Use \verb|pd.read_csv| to read \texttt{students.txt} into a DataFrame called \verb|students|.
              (Hint: the default separator is a comma, so you do not need to specify \verb|sep|.)
        \item Use \verb|pd.read_csv| to read \texttt{publications.txt} into a DataFrame called \verb|pubs|.
              (Hint: this file uses semicolons, so you must specify the correct separator.)
    \end{enumerate}

    \item Inspect the structure of the DataFrames.
    \begin{enumerate}
        \item Use \verb|students.info()| and \verb|pubs.info()| to view basic information.
        \item Display the first two rows and last two rows of each DataFrame using \verb|.head()| and \verb|.tail()|.
        \item Print the \verb|index|, \verb|dtypes| and \verb|values| attributes of \verb|students|.
    \end{enumerate}
\end{enumerate}

\section*{Part B: Cleaning and Transforming the Data}

\begin{enumerate}
\setcounter{enumi}{3}
    \item Clean up name formatting in the \verb|students| DataFrame.
    \begin{enumerate}
        \item Select the \verb|name| column as a Series using \verb|students['name']|.
        \item Use \verb|Series.str.replace| to:
        \begin{itemize}
            \item Replace any double spaces \verb|"  "| with a single space.
        \end{itemize}
        \item Use \verb|Series.str.capitalize| to convert each name to ``capitalized'' form
              (first letter uppercase, the rest lowercase).
        \item Assign the cleaned Series back to the \verb|students['name']| column.
        \item Verify the change by printing \verb|students.head()|.
    \end{enumerate}

    \item Handle missing GPA values.
    \begin{enumerate}
        \item Use \verb|students['gpa']| and \verb|Series.dtype| to check its current data type.
        \item Convert the \verb|gpa| column to numeric using \verb|pd.to_numeric|, and assign back.
        \item Use \verb|students.isnull()| to identify which rows have missing GPA values.
        \item Compute the mean GPA over all non-missing values using \verb|students['gpa'].mean()|.
        \item Use \verb|DataFrame.fillna| \emph{on the} \verb|gpa| column only to fill missing GPAs with the mean GPA.
        \item Confirm there are no remaining missing values in \verb|gpa|.
    \end{enumerate}

    \item Standardize text in the \verb|department| column (optional mini-cleaning).
    \begin{enumerate}
        \item Examine \verb|students['department'].unique()|.
        \item Use \verb|Series.str.replace| or \verb|Series.str.capitalize| as needed to ensure department names are consistently formatted.
    \end{enumerate}
\end{enumerate}

\section*{Part C: Selecting, Indexing and Filtering}

\begin{enumerate}
\setcounter{enumi}{6}
    \item Work with indexes.
    \begin{enumerate}
        \item Use \verb|students.set_index('student_id')| to create a new DataFrame
              \verb|students_by_id| where \verb|student_id| is the index.
        \item Display \verb|students_by_id.head()| and inspect \verb|students_by_id.index|.
        \item Access the row for student \verb|'S02'| using \verb|.loc|.
    \end{enumerate}

    \item Practice column selection.
    \begin{enumerate}
        \item From \verb|students|, select only the \verb|name| and \verb|gpa| columns using
              \verb|students[['name', 'gpa']]| and store in a new DataFrame \verb|name_gpa|.
        \item Display \verb|name_gpa.head()|.
    \end{enumerate}

    \item Filter rows based on conditions.
    \begin{enumerate}
        \item Using \verb|.loc| and Boolean conditions, select all active students in the
              \verb|Economics| department.
        \item From these students, select only the \verb|name|, \verb|start_year| and \verb|gpa| columns.
        \item Similarly, select all students with GPA strictly greater than 8.5.
    \end{enumerate}
\end{enumerate}

\section*{Part D: Working with the Publications Data}

\begin{enumerate}
\setcounter{enumi}{9}
    \item Basic inspection and typing.
    \begin{enumerate}
        \item Use \verb|pubs.head()| and \verb|pubs.tail()| to inspect the data.
        \item Check \verb|pubs.dtypes|.
        \item Ensure that the \verb|year| column is of an appropriate numeric type using \verb|astype| or \verb|pd.to_numeric|.
    \end{enumerate}

    \item Filtering and string methods.
    \begin{enumerate}
        \item Select only the journal publications using \verb|Series.str.contains| on the \verb|venue_type| column.
        \item Create a Series containing only the titles of journal publications.
        \item Use \verb|Series.str.capitalize| on the titles and inspect the result.
    \end{enumerate}

    \item Counting publications per student.
    \begin{enumerate}
        \item Use \verb|pubs['student_id']| to get the student ID Series.
        \item Create a pivot table counting the number of publications per student using
              \verb|DataFrame.pivot_table| with \verb|values='pub_id'|, \verb|index='student_id'|, and
              an appropriate aggregation function (e.g.\ counting).
        \item Reset the index of this pivot table using \verb|.reset_index()| so that \verb|student_id|
              becomes a column again.
    \end{enumerate}
\end{enumerate}

\section*{Part E: Combining Information}

\begin{enumerate}
\setcounter{enumi}{12}
    \item Compare students by publication counts and GPA.
    \begin{enumerate}
        \item Starting from the pivot table created in Part D, rename the column containing the counts
              to something meaningful like \verb|'pub_count'| using \verb|df.rename|.
        \item Create a copy of the \verb|students| DataFrame using \verb|students.copy()|.
        \item Set \verb|student_id| as the index of both DataFrames (the students copy and the pivot table)
              using \verb|.set_index('student_id')|.
        \item Use \verb|.loc| and \verb|DataFrame.T| (transpose) if helpful to manually align and inspect
              publication counts and GPAs for selected student IDs (e.g.\ \verb|'S01'|, \verb|'S02'|).
        \item Compute the mean GPA of students who have at least one publication (you may need to select
              only those student IDs present in the pivot table).
    \end{enumerate}

    \item Correlation exploration.
    \begin{enumerate}
        \item Construct a DataFrame that has, for each student present in the publications pivot table:
        \begin{itemize}
            \item their GPA,
            \item their number of publications.
        \end{itemize}
        \item Use \verb|df.corr()| on this DataFrame to compute the correlation matrix between GPA and publication count.
        \item Briefly interpret the resulting correlation value.
    \end{enumerate}
\end{enumerate}

\section*{Part F: Plotting and Exporting}

\begin{enumerate}
\setcounter{enumi}{14}
    \item Simple plots.
    \begin{enumerate}
        \item Use \verb|students['gpa'].plot()| to create a simple plot of GPA values (index on the x-axis).
        \item Use \verb|students.plot()| or \verb|DataFrame.plot| to create a plot that shows GPA by start year
              (you may wish to select only the relevant columns before plotting).
        \item Use the publication count DataFrame to create a bar plot (e.g.\ publication count by student ID).
    \end{enumerate}

    \item Exporting results.
    \begin{enumerate}
        \item Save the cleaned \verb|students| DataFrame to a CSV file called \texttt{students\_clean.csv}
              using \verb|df.to_csv|.
        \item Save the publication count per student to an Excel file called \texttt{pub\_counts.xlsx}
              using \verb|df.to_excel|.
    \end{enumerate}
\end{enumerate}

\section*{Part G: Additional Short Tasks}

\begin{enumerate}
\setcounter{enumi}{16}
    \item Using \verb|Series.unique|, list all distinct supervisor names.
    \item Using \verb|DataFrame.iterrows|, print (or construct a small Series) that maps each student
          name to their status (e.g.\ \verb|"alice smith" -> "active"|).
    \item Choose two Series (for example, two manually constructed Series of publication counts)
          and compare them using \verb|Series.equals|.
\end{enumerate}



\end{document}